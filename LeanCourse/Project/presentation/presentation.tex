\documentclass{beamer}
\usetheme{Berlin}
\usepackage{amsmath}
\usepackage{amsthm}
\usepackage{blindtext}
\usepackage{amsfonts}
\usepackage{graphicx}
\usepackage{tabto}

\title{Representation theory of finite groups}
\subtitle{Formalization project}
\author{Raphael Gaedtke, Paul Neumann}
\institute{University of Bonn}
\date{January 10, 2025}

\newcommand{\GL}{\text{GL}}

\begin{document}
\begin{frame}
\titlepage
\end{frame}

\begin{frame}
\frametitle{Outline}
\tableofcontents
\end{frame}

\section{Representation Theory}
\begin{frame}
\frametitle{Definition Representation}
\begin{definition}
For a group \(G\) and a field \(k\), a \textbf{representation} of \(G\) over \(k\) is a pair \((V, \rho)\) where \(V\) is a vector space over \(k\) and \(\rho: G\to \GL (V)\) is an action of \(G\) on \(V\).
\end{definition}
\pause
Convention: \(V\) has finite dimension, unless explicitly stated otherwise.
\begin{definition}
\(\dim (V)\) is the \textbf{dimension} or \textbf{degree} of \((V, \rho)\).
\end{definition}
\end{frame}

\note{
- intuition of a symmetry group
- this definition is somewhat restricted: Monoid instead of group, module instead of vector space is possible
- Representation theory is not restricted to groups
}

\section{Finite abelian groups}
\begin{frame}
\end{frame}

\section{Formalization}
\begin{frame}
\end{frame}

\section{Mathlib}
\begin{frame}
\end{frame}

\section{Future work}
\begin{frame}
\end{frame}

\end{document}